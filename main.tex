\documentclass[conference,compsoc]{IEEEtran}

\ifCLASSOPTIONcompsoc
  % IEEE Computer Society needs nocompress option
  % requires cite.sty v4.0 or later (November 2003)
  \usepackage[nocompress]{cite}
\else
  % normal IEEE
  \usepackage{cite}
\fi

\usepackage[pdftex]{graphicx}
% declare the path(s) where your graphic files are
% \graphicspath{{../pdf/}{../jpeg/}}
% and their extensions so you won't have to specify these with
% every instance of \includegraphics
% \DeclareGraphicsExtensions{.pdf,.jpeg,.png}

\usepackage{amsmath}

% Removido -- Claus. Estes dois pacotes sao necessarios?
%\usepackage[export]{adjustbox}
%\usepackage{pdfpages}% incluir PDFs, usado no apêndice

%\usepackage[brazil,american]{babel}
\usepackage[T1]{fontenc}
\usepackage{indentfirst}
%\usepackage{natbib}
\usepackage{xcolor,graphicx,url}
\usepackage[utf8]{inputenc}
\usepackage{amsmath}
\usepackage{graphicx}
\usepackage{url}
\usepackage{algorithm}
\usepackage{algorithmic}
\usepackage{graphicx}
\usepackage{lipsum}
% Note that the amsmath package sets \interdisplaylinepenalty to 10000
% thus preventing page breaks from occurring within multiline equations. Use:
%\interdisplaylinepenalty=2500
% after loading amsmath to restore such page breaks as IEEEtran.cls normally

%\usepackage{algorithmic}
%\usepackage{array}

% *** SUBFIGURE PACKAGES ***
%\ifCLASSOPTIONcompsoc
%  \usepackage[caption=false,font=footnotesize,labelfont=sf,textfont=sf]{subfig}
%\else
%  \usepackage[caption=false,font=footnotesize]{subfig}
%\fi

%\usepackage{fixltx2e}
% fixltx2e, the successor to the earlier fix2col.sty, was written by
% Frank Mittelbach and David Carlisle. This package corrects a few problems
% in the LaTeX2e kernel, the most notable of which is that in current

%\usepackage{stfloats}
% stfloats.sty was written by Sigitas Tolusis. This package gives LaTeX2e
% the ability to do double column floats at the bottom of the page as well
% as the top. (e.g., "\begin{figure*}[!b]" is not normally possible in
% LaTeX2e). It also provides a command:
%\fnbelowfloat
% to enable the placement of footnotes below bottom floats (the standard
% LaTeX2e kernel puts them above bottom floats). This is an invasive package

% *** PDF, URL AND HYPERLINK PACKAGES ***
%
%\usepackage{url}

% correct bad hyphenation here
\hyphenation{op-tical net-works semi-conduc-tor}


\begin{document}

\title{Improving the Generation of Earthquake Risk Models\\ Using
  Evolutionary Algorithms Tempered by Domain Knowledge}


% author names and affiliations
% use a multiple column layout for up to three different
% affiliations
\author{\IEEEauthorblockN{Yuri Lavinas}
\IEEEauthorblockA{University of Brasilia\\
Computer Science Department\\
yclavinas@gmail.com}
\and
\IEEEauthorblockN{Xiucai Ye}
\IEEEauthorblockA{University of Tsukuba\\
  Graduate School of SIE\\
yexiucai@mma.cs.tsukuba.ac.jp}
\and
\IEEEauthorblockN{Marcelo Ladeira}
\IEEEauthorblockA{University of Brasilia\\
Computer Science Department\\
mladeira@unb.br}
\and
\IEEEauthorblockN{Claus Aranha}
\IEEEauthorblockA{University of Tsukuba\\
Graduate School of SIE\\
caranha@cs.tsukuba.ac.jp}}

\maketitle

% As a general rule, do not put math, special symbols or citations
% in the abstract
\begin{abstract}

% File with the abstract
Earthquake Risk Models describe the risk of occurrence of seismic
events on a given area based on information such as past earthquakes
in nearby regions and the seismic properties of the area under study.
These models can be used to help to better understand earthquakes,
their patterns and their mechanisms.

In previous work, we showed that Genetic Algorithms (GA) could
generate risk models with the same degree of precision as the Relative
Intensity (RI) method, which is considered a benchmark for this
problem.  However, a few shortcomings were also defined in that
approach: (1) The representation of the model in the Genetic Algorithm
was too sparse, (2) Domain knowledge was not used to create the model,
and (3) The relationship between foreshocks and aftershocks were not 
taken into account.

In this work, we try to address these three concerns. We propose a new
representation of a seismic risk model to be used as the genome of the
Genetic Algorithm. We introduces a hybrid model that incorporates
seismic theories about earthquake distribution (such as the Omori-Utsu
formula). And we use clustering to filter the earthquake catalog in
order to remove earthquakes that are likely to be aftershocks before
generating the risk model.

We examine each of these changes through simulations using the catalog
of Japanese earthquakes between 2000 and 2010. According to our
results, clustering the earthquake catalog produces better models,
while the proposed changes to representation did not show such a clear
effect. These results allow us to draw recommendations for future
developments.


\end{abstract}

% no keywords

% For peerreview papers, this IEEEtran command inserts a page break and
% creates the second title. It will be ignored for other modes.
\IEEEpeerreviewmaketitle

% Main file with the paper
%%%%%%%%%%%%%%%%%%%%%%%%%%%%%%%%%%%%%%%%%%%%%%%%%%%%%%%%%%%%%%%%%%
\section{Introduction}\label{intro}

% Why are we studying this problem
Earthquakes can cause great damage to human society through soil
rupture, movement, tsunami, etc. Some recent earthquakes that
highlight this destructive potential are the great East Japan
Earthquake of 2011 (depicted in figure\ref{GreatEastJapan}), and the
April 2015 earthquake in Nepal. One important tool for the enactment
of policies that minimize the consequences of these events are
earthquake occurrence models (also called risk models). These models
can be used to identify patterns in the seismic mechanisms that
generate earthquakes, and are important to increase our understanding
of these events.


% \cite{ecta14} opening image. Would be better to have a clustered one.
\begin{figure}[]
\centering
\includegraphics[width=.45\textwidth]{img/earthquakes2011.png}
\caption{Seismic Activity in Eastern Japan in 2011. Each blue dot
  represents one earthquake}
\label{GreatEastJapan}
\end{figure}

% Context of our work
In our previous work~\cite{ecta14}, we proposed a way to generate
earthquake risk models using a standard Genetic Algorithm (here called
the GAModel). The GA model was shown to be competitive with the
Relative Intensity (RI) model, while not using any a-priori
information about the distribution of earthquake occurrences (See
section~\ref{sec:background} for a summary of the GAModel, along with
other relevant literature).

In this paper, we identify three key issues with the GAModel, and
propose adjustments to the algorithms that address these issues.

The first issue is that the genome representation used by GAModel has
too many parameters (over 2000 for regular cases). Even though a
majority of these parameters do not contribute for the accuracy of the
final risk model, the size of the search space implies a slower
optimization time. To address this issue, we propose a new genome
representation for an earthquake risk model, which we will call
``Reduced Representation''. In the Reduced Representation, we avoid
representing every single location in the area under study, and only
those locations with a minimal probability of earthquake are
represented as parameters in the evolutionary process. By reducing the
search space, this representation is expected to also increase the
convergence speed of the evolutionary optimization process.

% Second issue: Hybridization with domain knowledge
The second issue is that GAModel does not take into account any sort
of domain knowledge, such as the assumption that earthquakes cluster
in both time and space. Heuristic search methods such as Genetic
Algorithms usually benefit from the introduction of domain knowledge
to the search. Therefore, we propose a hybrid version of the GAModel
which incorporates seismic models of earthquake decay. This version
generates a model with a much smaller number of earthquakes than the
regular GAModel. For each earthquake in this model, a sequence of
aftershocks is generated using an adaptation of the Epidemic Type
Aftershock Sequence model (ETAS). We expect that this hybrid approach
will produce more accurate models.

The third issue is the examination of ``de-clustering'' effects in the
historical catalog used for generating the risk model. In seismology,
de-clustering refers to the act of identifying earthquakes as either
main-shocks or aftershocks, and removing all but the main shocks from
the catalog, which is considered the representative earthquake for the
group. Accordingly, a de-clustered earthquake catalog is considered to
be easier to study, given that the de-clustering process removes
redundant information~\cite{van2012seismicity}. In this work, we
generate the de-clustered catalog by grouping earthquakes in space and
time using spectral clustering~\cite{spectral_tutorial}.

These adaptions are described in detail in
section~\ref{sec:adaptations}. We compare the contributions of each
adaptation to the generation of models based on the earthquake catalog
of the Japanese arquipelago, between 2000 and 2010. The set up of this
experiment is described in section~\ref{sec:experiment},
and the main results are listed in section~\ref{sec:results}.

Our results indicate that clustering the earthquake catalog resulted
in a significant improvement to the precision of the models
generated. On the other hand, the new representation and the
hybridization did not seem to improve the results of our models. We
discuss the implications of these findings in
section~\ref{sec:discussion}.


\section{Background}
\label{sec:background}

An Earthquake Risk model states the probability of earthquake
occurrence on a defined area and time period. These models are often
based on past occurrence of earthquakes (historical catalogs).  They
can also make use of seismic properties of the area under study, such
as faults, terrain properties, etc.

The ``prediction'' of earthquakes is a polemic subject, and no
research so far has come close to suggesting that individual large
scale earthquakes can be predicted. On the other hand, there is value
on the study of earthquake mechanisms and the generation of
statistical models of earthquake risk~\cite{Nature1999}.

In our previous work~\cite{ecta14}, we use a Genetic Algorithm (GA) to
optimise an Earthquake Risk Model, which is described in the framework
proposed by the Collaboratory for the Study of Earthquake
Predictability (CSEP).

In the following subsections we describe both the CSEP framework and
the original GAModel. After that, we overview other relevant works
combining evolutionary approaches and the study of earthquakes.

\subsection{CSEP Forecast Framework}

The CSEP framework defines a model in reference to a geographical
region and a time period~\cite{zechar2010evaluating}. The geographical
region is divided in a grid, where each cell in the grid is called a
bin.

For example, in this paper we define the “Kanto” region as as the area
covered by latitude N34.8 to N37.05, and longitude E138.8 to
E141.05. This area is divided into 2025 bins (a grid of 45x45
squares).  Each bin has an area of approximately 25km$^2$

The model defines a number of expected earthquakes for each bin.  This
number must be a positive integer. A good model is one where the
number of estimated earthquakes in each bin corresponds to the actual
number of earthquakes that occurs in that bin during the target time
interval.


%%%%%%%%%%%%%%%%%%%%%%%%%%%%%%%%%%%%%%%%&&&&&&&&&&&&&&&&&&&&&&
\subsection{The GAModel}\label{sec:background:gamodel}

Using the CSEP framework described in the previous subsection, we
proposed the GAModel~\cite{ecta14}. The GAModel uses Genetic
Algorithms to generate an earthquake risk model based on earthquake
catalog data.

In the GAModel, each individual is a candidate Risk Model, which is
equivalent to a prediction $\Lambda = \{\lambda_1, \lambda_2, \ldots,
\lambda_n\}$ in the CSEP framework. The Genetic Algorithm will then
select the individuals based on the log-likelihood between the model
represented by the individual, and the catalog of earthquake
occurrences in the target location and time period.

\subsubsection*{Genome Representation}

In the GAModel, each individual is represented as real valued array,
where each element in the array is latitude/longitude bin in the
target area for the desired model. Each bin is associated to a real
value representing the earthquake risk in that location. For the
initial population, these values are drawn from an uniform
distribution between 0 and 1.

% TODO, maybe add the ``bins'' image from ECTA

During fitness evaluation, the risk value at each bin is converted to
an integer forecast, using a modification of the inverse Poisson
function depicted in algorithm~\ref{inversePoisson}. In this algorithm,
$x$ is the real value to be converted and $\mu$ is the mean of
earthquake observations across all bins in the catalog data.

\begin{algorithm}[H]\label{inversePoisson}
  \caption{Obtain a Poisson deviate from a $[0,1)$ value}
    \label{inversePoisson}
    \begin{algorithmic}
      \STATE $L \gets \exp{(-\mu)}, k \gets 1, prob \gets 1 * x$
      \WHILE{$prob > L$} 
      \STATE $k \gets k + 1$
      \STATE $prob \gets prob*x$
      \ENDWHILE
      \RETURN $k$
    \end{algorithmic}
\end{algorithm}

\subsubsection*{Fitness Function}

The GAModel uses the log-likelihood value between one individual
and the catalog data as its fitness function.

Let an individual $X = \{x_1, x_2, \ldots, x_N\}$, where $x_i$ is the
risk value associated with bin $i$, and $N$ is the total number of
bin. From $X$ we obtain the earthquake forecast $\Lambda =
\{\lambda_1, \lambda_2, \ldots, \lambda_N\}$ using
algorithm~\ref{inversePoisson}. This forecast is also the vector of
earthquake quantity expectations.

Let the set of earthquake occurrences from the catalog be $\Omega =
\{\omega_1, \omega_2, \ldots, \omega_N\}$. The calculation of the
log-likelihood value for the $\omega_i$ observation with a given
expectation $\lambda$ is defined as:

\begin{equation}
L(\omega_i|\lambda_i) = -\lambda_i + \omega_i\log\lambda_i - \log\omega_i!
\end{equation}

The joint probability is the product of the likelihood of each bin, so
the log likelihood $L(\Omega|\Lambda)$ for the entire vector is the
sum of $L(\omega_i|\lambda_i)$ for every bin $i$:

\begin{equation}
L(\Omega|\Lambda) = \sum_{i=1}^{n}L(\omega_i|\lambda_i) \\ =
\sum_{i=1}^{n} -\lambda_i + \omega_i\log\lambda_i - \log\omega_i!
\end{equation}

Finally, this value is calculated for each year that composes the
training data, using the ``time slices'' method described
in~\cite{ecta14}.


\subsubsection*{Evolutionary Operators}\label{gaOperators}

GAModel uses One Point Crossover, Polynomial Bounded Mutation,
Tournament Selection and Elitism. The chance for using crossover and
mutation operators is tested independently for each individual in the
new population.

Values for the parameters for these operators are listed in
Table~\ref{GAParameters5.1}. These values are generally the same as
used in~\cite{ecta14}.

\begin{table}[H]
  \caption{Parameters used in the GAModel}
  \label{GAParameters5.1}
  \begin{center}
    \begin{tabular}{|l|r|}
      \hline
      Population Size & 500\\
      Generation Number & 100\\
      Elite Size & 1\\
      Tournament Size & 3\\
      Crossover Chance & 0.9\\
      Mutation Chance (individual) & 0.1\\
      Polynomial Bounded parameters & eta = 1, low = 0, up = 1\\
      \hline    
    \end{tabular}
  \end{center}
\end{table}

\subsection{Related Literature}

The usage of Evolutionary Computation (EC) in the field of earthquake
risk models is somewhat sporadic. Zhang and Wang~\cite{Zhang2012} used
Genetic Algorithms to fine tune an Artificial Neural Network (ANN) and
used this system to produce a forecast model. Zhou and
Zu~\cite{Feiyan2014} also proposed a combination of ANN and EC, but
their system forecasts only the magnitude parameter of
earthquakes. Sadat, in~\cite{sadat2015application}, used ANN and GA to
predict the magnitude of the earthquakes in North Iran.

%Fault Model parameters
There are more works when we discuss EC methods and estimation of
parameter values in seismological models. Nicknam et
al.~\cite{Nicknam2010} simulated some components from a seismogram
station and predicted seismograms for other stations. They combined
the Empirical Green’s Function (EGF) with GA. Kennett and
Sambridge~\cite{Kennett1992} used GA and associated teleseisms
procedures to determine the Fault Model parameters of an
earthquake. By doing so, they demonstrated that non-linear inversion
can be achieved for teleseismic problems without any calculation of
waves travel times.

%PGA
Another popular approach is to use EC methods do calculate the Peak
Ground Acceleration (PGA) parameter. The works done by Kerh et
al.~\cite{Kerh2010, Kerh2015} are a combination of ANN and GA to
estimate or predict PGA in Taiwan. Their goal was to decide which
areas may be considered potentially hazardous areas and they focused
on urban areas. They also stated that PGA is inversely proportional to
epicentre distance. Cabalar and Cevik~\cite{Cabalar2009} work also
aimed to predict the PGA, but their work uses genetic programming (GP)
and use strong-ground-motion data from Turkey.

Jafarian et al.~\cite{jafarian2010empirical}, used GP to develop an
empirical predictive equation $v_max/a_max$ ratio of the shallow
crustal strong ground motions recorded at free field sites. They found
a relation between the $v_max/a_max$ and the earthquake magnitude and
the source-to-site distance.
 
Ramos and Vázques~\cite{Ramos2011} used Genetic Algorithms to decide
the location of sensing stations. In this work they achieved, in
general, better results with the GA method when compared with the
Seismic Alert System (SAS) method and a greedy algorithm
method. Saeidian et al.~\cite{saeidian2016evaluation} work also based
on the same idea of locating sensing stations. They do a comparison in
performance between the GA and Bees Algorithm (BA) to decide which of
those techniques would perform better when choosing the location of
sensing stations. He found out that the GA was faster than the BA.

Huda and Santosa \cite{ijse5762} published a paper in which the goal
was to find, via GA, the speed of the waves P and S in the mantle and
in the earth crust. P waves are indicated as the first fault found in
seismological data and S waves are the changes caused in the phase of
a P wave~\cite{ijse5762}. This work aimed to obtain a structure of the
Japanese underground.
 % Earthquake model, CSEP, references (including italy)
\section{Proposed Changes}
\label{sec:adaptations}

In this work, we propose three improvements to the GAModel: A reduced
genome representation, Hybridisation with the ETAS empirical model,
and the clustering of the earthquake catalog. Each of these changes
are described below.

%%%%%%%%%%%%%%%%%%%%%%%%%%%%%%%%%%%%%%%%%%%%%%%%%%55
\subsection{Reduced Genome Representation}

In the GAModel, problem is represented as a vector $X$ where each bin
corresponds to an element in the vector. As the number of bins in a
region numbers into the thousands, this representation leads to a huge
search space to be explored.

We have observed that in many cases, the vector of catalog
earthquakes is sparse. In other words, most of the elements of $X$
will be zero or close to it. To use this fact to decrease the
search space, we propose a ``reduced'' representation of a risk Model.

A summary of the reduced representation is as follows: First, before
initialising the Genetic Algorithm, we estimate the expected total
number of earthquakes in the model based on the past data. Then, this
value is used as the total number of earthquakes to be added to the
model. The reduced representation will be a vector of bin coordinates
for each of these earthquakes, representing their position inside the
target area. This is much smaller than a representation including each
bin as an element.


\subsubsection*{Implementation}

The reduced representation is a vector $V$ of ordered pairs. The first
element of this pair is the integer index that identify a bin in the
model. The second element of the pair is the number of earthquake
occurrences estimated for this bin.

The size of the vector $V$ is calculated as the number of bins in the
historical catalog that contain at least one earthquake. For each
element in $V$, the bin index and the estimated number of occurrences
are drawn randomly from a uniform distribution.

To generate a model from the reduced representation, we need to go two
intermediate steps. The first one is to transform the reduced
representation into a regular representation. This is achieved by
copying the estimated value of an element to the bin indicated by
stored index for that element. Bins that are not indicated by any
element in the vector are set to zero estimated earthquakes.

The second step is to apply the inverse Poisson on the estimated
values to retrieve the number of earthquakes, as described in
algorithm~\ref{inversePoisson}.

\subsubsection*{Operators}

The reduced representation can use the same one point crossover as the
GAModel, but a different mutation operation is required. The mutation
operator works by selecting one element in the vector $V$, and drawing
new values for the index and the estimation parameter from a uniform
distribution.


%%%%%%%%%%%%%%%%%%%%%%%%%%%%%%%%%%%%%%%%%%%%%%%%%%%%%%%%%%%%%
\subsection{Hybridisation with ETAS}

% Motivation
The GAModel produces risk models without using any sort of domain
knowledge, other than the difference between the individual being
evaluated and the earthquake catalog data.

However, one simple observation that could be added to the GAModel is
that earthquakes cluster in space and time. Large earthquakes are
usually followed by a wave of smaller earthquakes, these pairings
being commonly known as \emph{mainshocks} and
\emph{aftershocks}~\cite{schorlemmer2010first}.

To include this idea into the GAModel, we modify the process which
generates a Model from an individual. In this modified process, one
individual will only produce mainshocks into the model, afterwards the
aftershock are derived from the mainshocks, using empirical seismic laws
such as the \emph{modified Omori Law}.

We define this hybridisation between empirical seismic laws and the
GAModel as the \emph{EMP-GA}. Below, we detail the implementation of
both steps.

%%%%%%%%%%%%%%%%%%%%%%%%%%%%%%%%%%%%%%%%%%%%%%%%%%%%%%%%%%%%%%%%%
\subsubsection*{Implementation}

The EMP-GA generates models with mainshocks and aftershocks following
a two-step procedure.

In the first step, we use the GAModel to generate a set of mainshock
earthquakes, which we will refer to as \emph{synthetic mainshock
  data}. In the second step, we use seismic empirical equations to
obtain the aftershocks from the synthetic mainshock data, and add them
to the model.

The process we use to generate aftershocks from the synthetic
mainshock data is inspired by the space-time epidemic-type aftershock
sequence (ETAS). The total number of earthquakes in a bin is given as
% TODO: Sentar com o Yuri e pedir para ele me explicar a
% implementacao/razao desta equacao detalhe por detalhe.
\begin{equation}\label{emp-model}
 \Lambda(t,x,y|\Upsilon_t) = [\mu(x,y) + \displaystyle\sum_{t_i \in t}
   K(M_i)g(t-t_i)P(x,y)]J(M).
\end{equation}
In this equation, $t$ is the target time for the model, $x,y$ are
latitude/longitude coordinates within the target area, $\Upsilon_t$ is
the number of mainshocks derived in the first step, $\mu(x,y)$ is the
expected number of earthquakes at the $(x,y)$ bin, $t_i$ is a time
interval within $t$, $K(M_i)$ is the total amount of triggered events,
$g(\Delta t)$ is the probability density form of the modified Omori
law, $P(x,y)$ is a function that distributes aftershocks in space
nearby the mainshock, and $J(M)$ is the ETAS simulated magnitude. Let
us explain each of these components below.

\subsubsection*{Omori's Law and Triggered Events}

The Omori law, which is considered to be an empirical seismic formula
which has withstood the test of
time~\cite{utsu1995centenary,omori1895after}, is a power law that
relates the magnitude of an earthquake with the decay of aftershock
activity over time. It can estimate the number of aftershocks based on
the mainshocks in the synthetic data generated by one individual in
the EMPGA. For this approach, we use the probability density function
(PDF) form of the modified Omori law~\cite{zhuang2004analyzing},
defined as

\begin{equation}\label{omori}
  g(\Delta t)= \frac{(p-1)}{c(1+ \frac{t}{c})^{-p}}.
\end{equation}

In this equation, $p$ and $c$ are
constants. Utsu~\cite{utsu1995centenary}, summarise the studies of
this formula for the Japan case, and describe a range for these
variables using the Davidon-Fletcher-Powell optimisation
procedure. These ranges, used in ETAS, are 0.9 to 1.4 for $p$, and
0.003 and 0.3 for $c$.

Also, $\Delta t$ is the time interval for how long a mainshock may
influence or cause an aftershock. According to
Yamanaka~\cite{yamanaka1990scaling}, this value must be chosen
carefully, for a value too short will lead to a small number of
aftershocks, while a value too long might confound aftershocks and
background activity. His work suggests the values $p = 1.3$, $c =
0.003$, and $\Delta t = 30$ days, which we use in this paper.

The total amount of events triggered by a mainshock is represented in
equation~\ref{emp-model} as $K(M_i)$. To calculate this value,
we count the number of aftershocks within a given area $A$ from
the mainshock, using the formula

\begin{equation}\label{triggered}
 K(M_i) = A\ exp([\alpha(M_i-M_c)]).
\end{equation}

Where $M_c = 3.0$ is the magnitude threshold and $\alpha(M)$ is defined
as the inverse of the magnitude, according to
Ogata~\cite{ogata2006space}. The area $A$ is obtained using the
equation from Yamanaka~\cite{yamanaka1990scaling}

\begin{equation}
A = e^{(1.02M -4)}.
\end{equation}

Using the number of triggered events per magnitude $K(Mi)$, and the
Modified Omori PDF $g(t)$, it is possible to calculate the total
number of earthquakes generated from a mainshock, by iterating
over $t_i$:
\begin{equation}
\displaystyle\sum_{t_i \in t} K(M_i)g(t-t_i)
\end{equation}

% TODO: From this description, it does not seem that P(x,y) is actually
% part of equation emp-model: double check with Yuri.
The resulting aftershocks need to be spread on bins near the mainshock
position. The $P(x,y)$ component of equation~\ref{emp-model} fills
this role. It calculates the position of the aftershocks based on the
position of the original mainshock. It simply places each aftershock
either north, south, east or west of the mainshock, getting further
from the origin after each iteration, until there are no more events
to be placed.

%% This equation is not clear -- where is the component that place
%% Aftershocks further away from the origin?
%\begin{subequations}
%\begin{gather*}
%        model[x+y] = (aftershocks-[model[x]-2*x])/4;\\
%        model[x-y] = (aftershocks-[model[x]-2*x])/4;\\
%        model[x-y*row] = (aftershocks-[model[x]-2*x])/4;\\
%        model[x+y*row] = (aftershocks-[model[x]-2*x])/4
%\end{gather*}
%\end{subequations}

% TODO: but WHAT is J(M)?
Finally, $J(M)$ is obtained by using the function \emph{etasim}, from
the SAPP \textit{R} package~\cite{webSapp} that simulates magnitude by
Gutenberg-Richter’s Law.

The above equations are put together in algorithm~\ref{algoEquations}.

\begin{algorithm}[H]\label{algoEquations}
  \caption{Aftershock distribution from empirical laws}
  \begin{algorithmic}
    \STATE FOR EACH BIN:
    \IF {Number of earthquakes in bin > 12} % Why?
    \STATE {Reduce number of earthquakes in bin to 12}
    \ENDIF
    \STATE aftershocks = 0
    \STATE magnitude values for earthquakes in bin = J(M)
    % What does this mean?
    % \STATE model  = attributeMagnitudeToEarthquake(model, J(M) )
    % \STATE magnitudes = getMagnitudeMainshock(model)
    
    \FOR{magnitude in magnitudes} 
    \FOR{t in time} 
    \STATE aftershocks += g(t)*K(magnitude)
    \ENDFOR
    \ENDFOR
    % Need a more detailed algorithm for P(x,y)
    \STATE Use P(x,y) to distribute aftershocks to neighbour bins
  \end{algorithmic}
\end{algorithm}

% TODO: Probably needs a longer explanation of ETAS in the
% Bibliography section.

%%%%%%%%%%%%%%%%%%%%%%%%%%%%%%%%%%%%%%%%%%%%%%%%%%%%%%%%%%%%%%%%%5
\subsection{Clustering the Catalog Data}

The third adaptation to the GAModel that we study in this paper is the
clustering of the earthquake catalog data using Spectral
Clustering. Unlike the two adaptations described beforehand, this one
does not require any change on the algorithm itself, happening instead
as a data pre-processing step when building the model. After the
catalog data is pre-processed, the Genetic Algorithm is applied
normally, using the de-clustered data for fitness evaluation.

This pre-processing step aims to remove redundant information from the
catalog, by clustering together earthquakes which are closely related
in a mainshock/aftershock relationship~\cite{van2012seismicity}.

However, because it is difficult to determine exactly when two
earthquakes should be clustered together, we choose a non-supervised
method, Spectral Clustering, to generate the clusters.

Spectral Clustering involves constructing a similarity matrix of the
elements to be clustered, finding the k-Nearest-Neighbours graph (KNN)
based on the similarity matrix, calculating the Laplacian matrix of
the KNN graph, and performing k-means clustering on the eigenvectors
of this matrix.

One of the main characteristics of Spectral Clustering that make it
interesting for this problem is that it can be very computationally
efficient~\cite{Ye2016}. This is very important for the clustering of
earthquake data, since each data set can contain tens of thousands of
earthquakes.

\subsubsection*{Spectral Clustering Implementation}

Let the earthquake catalog data be represented as a vector $X = \{x_1,
x_2, \ldots, x_n\ in \Re_d\}$, where $n$ is the number of earthquakes
in the catalog, $d$ is the number of attributes that characterise an
earthquake in the catalog, and $K$ is the desired number of
clusters. The clusters are calculated following
algorithm~\ref{spectralclustering}

% TODO: Probably needs to include how to calculate KNN
\begin{algorithm}[H]\label{spectralclustering}
  \caption{Spectral Clustering}
  \begin{algorithmic}
    \STATE{Construct the similarity matrix S}
    \FOR{i in $X$}
    \FOR{j in $X$}
    \IF{$i$ and $j$ are connected in the KNN graph}
    \STATE{$s_{i,j} = \exp(-||x_i-x_j||^2/2\sigma^2$}
    \ELSE
    \STATE{$s_{i,j} = 0$}
    \ENDIF
    \ENDFOR
    \ENDFOR
    \STATE Matrix $D = n x n$ diagonal matrix where $d_{i,i} = \sum^n_{j=1}s_{ij}$
    \STATE Compute Matrix $L = D - S$
    \STATE Compute $K$ smallest eigenvectors of $L$
    \STATE Compute matrix $V = (v_{ij})_{nxK}$, using these eigenvactors as columns.
    \STATE Compute matrix $U = (u_{ij})_{nxK}$, normalising the rows
    of $V$ such as $u_{i,j} = v_{i,j}/\sqrt{\sum_jv^2_{ij}}$
    \STATE Let each row in $U$ represent a data point, and cluster
    these points using k-means
    \FOR{each point $x_i$ in $X$}
    \STATE Assign the cluster of $u_i$ to $x_i$
    \ENDFOR
  \end{algorithmic}
\end{algorithm}

In this algorithm, $||x_i-x_j||$ is the Euclidean distance between
data points $x_i,x_j$. We use the number of nearest neighbours equal to
five, and $\sigma$, the kernel parameter, equal to $100$.

% TODO: Minutes is divided by 1000 to make the values smaller
We cluster the earthquakes based on their latitude, longitude, time
(in minutes), and depth. By observing the distributions of the
eigenvectors, we defined the weight of each dimension in the algorithm:

\begin{itemize}
\item latitude and longitude: 150
\item time: 7
\item Depth: 0.5
\end{itemize}


% TODO: Change the tone of the introduction, talk about exploring the
% TODO: proposed changes

\section{Experiment Design}
\label{sec:experiment}

% Goal of the experiments
%% We want to explore the proposed modifications, so we test the 
%% Effects of parameters in these modifications, and the effects of
%% modifications in the model accuracy




\subsection{Experiment Design}

% - Goals/List of experiments
%% - Analysis of parameters (Clustering/Others)
%% - Comparison of the influence of the modifications



% - How we execute the experiments
%% - For each of the proposed analysis, we run
%% the evolutionary algorithm to generate a 
%% risk model, and we compare the resulting models

% - How we analyze the experiments
%% Comparison is done as...


\subsubsection*{Claus' Extracts}

Statistical analysis: ANOVA on the populational means

Tukey HSD for significant differences on ANOVA

- 

\subsubsection*{Old Text}

The first experiment was made to compare the all the models proposed
with each other and to discover which method would achieve higher
log-likelihood values. We created some scenarios (space/time regions),
and we applied the methods for for the regions of Kanto, Kansai,
Touhoku and East Japan for a given year (2005-2010) with earthquakes
with depth lesser than 100km. We also used 3 kinds of catalogues with
the minimum magnitude of 3.0: the JMA and the declustered catalogues
form the Window method and the SLC method.

We compared the means of the models log-likelihood values using the
ANOVA test. If a group of variables considered for the ANOVA test
showed no statistically significant difference, we applied the Paired
Student t-test, in the case all groups showed statistically
significant difference, the Tukey HSD methodology analysis was used.


%% ????
The second experiment was made a to compare how the magnitude of the
earthquakes influence the models generated. We used the same scenarios
from the first experiment. We split the models obtained from these
scenarios into slices composed of earthquakes that have magnitude in a
given magnitude interval. We calculated the log-likelihood of these
slices-models and applied the ANOVA test and the Tukey HSD to compared
them.






%%%%%%%%%%%%%%%%%%%%%%%%%%%%%%%%%%%%%%%%%%%%%%%%%%%%%%%%%%%%%%%%%%%%%%%5
\subsection{Data Sets}

The earthquake catalog from the Japanese arquipelago were obtained
from the \emph{Japan Meteorological Agency} (JMA) webpage.

% - How we obtained the data sets

% - Explanation of the data sets (locations)

% - Parameters of the data sets


The goal of this research is to find existing patterns in the
occurrence of earthquakes. For that it is essential to access trustful
data and to explore its details. From the {\it Japan Meteorological
  Agency} web page we obtained earthquake data about earthquakes in
Japan. In this data there are information about earthquakes that
happened in or nearby Japan, with the variables: time of the
occurrence, magnitude, latitude and longitude and epicentre depth, for
the years of 2000 to 2013.

During the preprocessing phase, we discovered a higher number of
occurrences of earthquakes during the year of 2011, when a 9.0 $M_w$
earthquake happened, see Section~\ref{chapter1}. This earthquake
triggered too many after called aftershocks in all Japan. It is
considered that big earthquakes may cause others
earthquakes~\cite{zhuang2004analyzing}. In~Figure \ref{ocorrenciasAno}
it is possible to visualise a great number of earthquakes for the year
of 2011. Because of this abnormal behaviour and because we decided to
focus on more stable occurrences, we limited the training base to
earthquakes until 2010.

\begin{figure}[]
\centering
\includegraphics[scale=0.5]{img/ocorrenciasAno.png}
\caption{Amount of earthquake by year.}
\label{ocorrenciasAno}
\end{figure}


Based on the statement done before and considering that we want
earthquakes that follow more stable patterns, we selected the ones
that happened in land areas or very shallow sea areas, with maximum
depth of 100km.

For the experiments, the data was changed into slices for every
year. Each slice is as follows: if the base contains data about a time
interval of 10 years, it will be split in 10 slices.

We also selected some sub-areas in Japan to better extract and
understand earthquakes characteristics and patterns. Those areas are
Kanto, Kansai, Touhoku and East Japan. The Figure~\ref{alljapan} shows
how we defined them.

\begin{figure}
	\centering
	\includegraphics[scale=0.25]{img/alljapan.png}
	\caption{Japan and the areas used in this studied.}
	\label{alljapan}
\end{figure}


The regions are described as follows:

\textbf{Kanto} Kanto is the region around Tokyo. It is an area with
high seismologic activity during the years we studied. Its coordinates
are 34.8 North, 138.8 West, with 2025 bins. Each bin covers an area of
approximately 25km$^2$.

\textbf{Kansai} Kansai is the region that includes Kyoto, Osaka and
many others historical cities. In this area, rather than Kanto area,
there is a small seismic activity. Its coordinates are 34 North, 134.5
West, with 1600 bins. Each bin covers an area of approximately
25km$^2$.

\textbf{Touhoku} Touhoku is the region in the North of the main
Japanese island. It has some clusters of seismic activities during the
years we studied. Its coordinates are 37.8 North, 139.8 West, with 800
bins. Each bin covers an area of approximately 100km$^2$.

\textbf{East Japan} Is the region that is related with the east coast
of Japan. It is the most different area, because it has earthquakes
that happened both in land or in the sea. It was in this region that
the 9.0 $M_w$ earthquake happened. Its coordinates are 37 North, 140
West, with 1600 bins. Each bin covers an area of approximately
100km$^2$.

\subsubsection{Depth Histogram of Earthquakes}


The patterns of earthquakes are dependent of the epicentre. We wanted
to explore the relation between the depth of the earthquakes and how
would our models behave on those situations.

In Figure~\ref{histogramQuakes}, it is possible to understand that
most of the earthquakes happened with depths smaller or equal to 100
km. The earthquakes deeper than 100 km are fewer and more distant, as
it is in the same Figure.\\

\begin{figure}[]
	\centering
	\includegraphics[scale=0.15]{img/detphsNew.png}
	\caption{Depth Histogram of earthquakes.}
	\label{histogramQuakes}
\end{figure}

The reason we decided to groups as: earthquakes with depth until 25
km, until 60 km or until 100 km. This is because shallow earthquakes
are considered to be more independent
earthquakes~\cite{yamanaka1990scaling}.

\begin{figure}
	\centering
	\includegraphics[scale=0.35]{img/Magnitude2003Kanto.png}
	\caption{Histogram of earthquakes stronger than 3.0 in Kanto}
	\label{quakesKanto}
\end{figure}

\section{Results}
\label{sec:results}

Table~\ref{tab:loglikelihood} summarizes the results of the
experiments comparing the eight combinations over the different scenarios. 

\begin{table}
  \caption{Log Likelihood Values for each scenario and
    combination. Higher values correspond to better models. Values in
    bold are the two highest log likelihood for a scenario. Each value
    is the average of 10 runs.}
  \label{tab:loglikelihood}
  \begin{tabular}{|c|c|c|c|c|c|c|c|c|}
    \hline
    & \multicolumn{4}{|c|}{No Pre-processing} & \multicolumn{4}{|c|}{Spectral Clustering}\\
    Scenario & GA & Red & EMP & EMP-Red & GA & Red & EMP & EMP-Red\\
    \hline
    Kanto 2005 & -2291.4 & -2354.1 & -2345.1 & -2557.8 & {\bf-2202.7} & -2233.3 & {\bf-2203.7} & -2355.0\\
    Kanto 2006 & -2269.2 & -2317.3 & -2350.7 & -2520.0 & {\bf-2173.3} & -2203.1 & {\bf-2175.8} & -2313.5\\
    Kanto 2007 & -2204.9 & -2235.7 & -2293.3 & -2449.5 & {\bf-2104.1} & -2125.0 & {\bf-2110.3} & -2213.9\\
    Kanto 2008 & -2203.0 & -2273.2 & -2277.7 & -2501.0 & {\bf-2097.9} & -2124.3 & {\bf-2010.3} & -2245.7\\
    Kanto 2009 & -2375.6 & -2418.5 & -2463.6 & -2630.7 & {\bf-2279.1} & -2299.1 & {\bf-2282.4} & -2382.0\\
    Kanto 2010 & -2203.6 & -2296.6 & -2294.9 & -2534.4 & {\bf-2099.5} & -2125.0 & {\bf-2104.0} & -2249.8\\
    \hline
    EastJapan 2005 &-2442.8&-2394.9&-2633.6&-2588.4& {\bf-2099.6} & {\bf-2150.2} & -2177.4 & -2300.6 \\
    EastJapan 2006 &-2211.1&-2191.7&-2408.9&-2390.9&{\bf-1896.7}&{\bf-1960.4}&-1965.7&-2131.8\\
    EastJapan 2007 &-2112.2&-2100.5&-2305.1&-2294.9&{\bf-1821.9}&{\bf-1889.4}&-1914.4&-2070.0\\
    EastJapan 2008 &-4139.7&-4288.6&-4301.3&-4424.8&{\bf-3942.5}&{\bf-3989.1}&-4034.9&-4156.8\\
    EastJapan 2009 &-2281.2&-2221.2&-2498.9&-2416.5&{\bf-1948.5}&{\bf-1087.4}&-2043.7&-2164.5\\
    EastJapan 2010 &-2577.7&-2579.1&-2783.9&-2783.9&{\bf-2232.7}&{\bf-2291.3}&-2296.9&-2455.2\\
    \hline
  \end{tabular}
\end{table}

%% ANOVA ANALYSIS

To better understand these results, we perform an ANOVA analysis...

* Anova in all areas

Because the anova indicated a significant difference, we use Tukey's
HSD to see which combination showed this difference ...

* HSD Tukey against GAModel

%% HEAT MAP

To get a better intuition about what these results mean in concrete
terms, we show a selection of the actual models... (explain heat map)

* Heat Maps

\section{Discussion}
\label{sec:discussion}

Clustering is good. It is okay if Reduced GA is not worse, because
"lower search space"

\section{Conclusion}
\label{sec:conclusion}

In this paper we proposed three improvements to a Genetic Algorithm
for earthquake risk modeling. Pre-processing the data using Spectral
Clustering showed the greatest improvement, while simplifying the
representation reduced the number of parameters in the model without
reducing its quality. Using seismic equations, though showed less
successful.

These results show that there is a need for further ideas for the use
of meta-heuristics in modeling seismic processes. For example, the
Spectral Clustering step currently uses a trivial distance measure;
using a more appropriate modeling of space-time clustering of
earthquakes is one approach that shows promise.

Finally, many choices in this work regarding the target regions and
data filters aimed to focus our effort in ``regular'' scenarios of
earthquake occurrence. While we feel this is an appropriate decision
for an early work, we aim to remove these constraints from the data
set in the future, in order to see how the difficulty of the problem
affects the approaches used.


\section*{Acknowledgments}
The authors would like to thank the Japan Meteorological Agency for
making available the earthquake catalog used in this study.


\bibliographystyle{IEEEtran}
\bibliography{bib}

\end{document}


