%%%%%%%%%%%%%%%%%%%%%%%%%%%%%%%%%%%%%%%%%%%%%%%%%%%%%%%%%%%%%%%%%%
\section{Introduction}\label{intro}

% Why are we studying this problem
Earthquakes can cause great damage to human society through soil
rupture, movement, tsunami, etc. Some recent earthquakes that
highlight this destructive potential are the great East Japan
Earthquake of 2011 (depicted in figure\ref{GreatEastJapan}), and the
April 2015 earthquake in Nepal. One important tool for the enactment
of policies that minimize the consequences of these events are
earthquake occurrence models (also called risk models). These models
can be used to identify patterns in the seismic mechanisms that
generate earthquakes, and are important to increase our understanding
of these events.


% \cite{ecta14} opening image. Would be better to have a clustered one.
\begin{figure}[]
\centering
\includegraphics[width=.45\textwidth]{img/earthquakes2011.png}
\caption{Seismic Activity in Eastern Japan in 2011. Each blue dot
  represents one earthquake}
\label{GreatEastJapan}
\end{figure}

% Gabriel's image, does not look very good.
%\begin{figure}[]
%	\centering
%	\includegraphics[width=.55\textwidth]{img/slc_leste.png}
%	\caption{Seismic Activity in Eastern Japan during the years of 2003 to 2011.}
%	\label{GreatEastJapan}
%\end{figure}

% Context of our work
In our previous work~\cite{ecta14}, we proposed a way to generate
earthquake risk models using a standard Genetic Algorithm (here called
the GAModel). The GA model was shown to be competitive with the
Relative Intensity (RI) model, while not using any a-priori
information about the distribution of earthquake occurrences. However,
we have identified three key issues with the GAModel. Addressing
these issues will be the focus of this paper.

% Ideia and contribution: First issue, representation was too sparse.
% TODO: the non contribution of the parameters is evidenced by the
% random noise of the parameters. It might be good to approach this in
% section 2.
The first issue is that the genome representation used by GAModel has
too many parameters (over 2000 for regular cases). Even though a
majority of these parameters do not contribute for the accuracy of the
final risk model, the size of the search space implies a slower
optimization time. To address this issue, we propose a new genome
representation for an earthquake risk model, which we call
ReducedGAModel. In the ReducedGAModel, only areas with minimal
probability of an earthquake are represented as parameters in the
evolutionary process. By reducing the search space, this
representation is expected to also increase the convergence speed of
the evolutionary optimization process.

% Second issue: Hybridization with domain knowledge
The second issue is that GAModel does not take into account any sort
of domain knowledge, such as the assumption that earthquakes cluster
in both time and space. Heuristic search methods such as Genetic
Algorithms usually benefit from the introduction of domain knowledge
to the search. Therefore, we propose a hybrid version of the GAModel
which incorporates seismic models of earthquake decay. This version,
named Emp-GAModel, generates a model with a much smaller number of
earthquakes than the regular GAModel. For each earthquake in this
model, a sequence of aftershocks is generated using an adaptation of
the Epidemic Type Aftershock Sequence model (ETAS). We expect that
this hybrid approach will produce more accurate models.

% Third issue, ``clustering is a good thing''


% Summary of our experiments (EMPGA x ReducedGA, Clustering)
% Summary of our results

% End of current changes
%%%%%%%%%%%%%%%%%%%%%%%%%%%%%%%%%%%%%%%%%%%%%%%%%%%%%%%%%%%%%%%%%%%

%% TODO:
% Second issue, we do not use domain knowledge. Risk is assigned randomly
% Solution, emp-GAmodel, where we generate only ``main shocks'' first,
% and generate aftershocks from the main shocks using ETAS.

% In this paper we will explore EMPGA and ReducedGA separately, as well as 
% In concert (ADD new 2x2 figure/table here)

% Third issue, clustering. ``it is known'' that clustering earthquakes
% and removing aftershocks may make earthquake catalogs easier to analyse
% In this sense, we use two different clustering techniques, and 
% Test the above GAmodel variations in the clustered catalog as well

% The results show which approaches improve on the results of the 
% Original GAModel, and which do not offer any significant contribution

%%%%%%%%%%%%%%%%%%%%%%% Untouched text %%%%%%%%%%%%%%%%%%%%%%%%%
The other idea is based on the assumption that earthquakes cluster in
both space and time, and we want incorporate in the Genetic Algorithm
technique (GA) some geophysical knowledge. It is a hybridisation of
the models generated with GA some empirical laws, such as the modified
Omori law. First, the background intensity (the independent
earthquakes or mainshocks), which is a function of the space, is
forecast using the GA. Then, we use some empirical laws to obtain the
dependent earthquakes (aftershocks) for a specific time interval.

The Emp-GAModel is a method proposed that incorporates some
geophysical knowledge. It is a hybridization of the models generated
by the GAModel with the these empirical laws, see Section
\ref{Models}.

Finally, there is the Emp-ReducedGAModel. This method is a combination
of the two ideas. Therefore, it also performs a hybridization of
models with the group of empirical law. Though, for this method, the
models are generated by the ReducedGAModel.

The forecast models produced by those methods and the ones produced by
the GAModel were all analyzed by their log-likelihood values
calculated as suggested by the Regional Earthquake Likelihood Model
(RELM)~\cite{schorlemmer2007earthquake}.

For developing the methods and to be able to compare them we used the
earthquake catalog from the Japanese Meteorological Agency (JMA),
using event data from 2005 to 2010.

This paper is organized as: in Section \ref{estadoArte}reviews
applications of Evolutionary Computation in the context of seismology
research. The next Section,Section \ref{Models}, we give a details of
each of the forecast proposed covering the Collaboratory for the Study
of Earthquake Predictability (CSEP) framework and the empirical
laws. In Section \ref{Tests}, we give the description of the tests
proposed in~\cite{Schorlemmer2007}. After that, in \ref{exp}, we
define the target areas used for the experiment and the data from the
JMA; we clarify the design followed during the experiments and how we
compared the forecast models derived from our methods. Finally, we
show the results and conclude this work in~\ref{Results} and
\ref{Conclusions}.
