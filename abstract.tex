Earthquake Risk Models describe the risk of occurence of seismic
events on a given area based on information such as past earthquakes
in nearby regions, and the seismic properties of the area under study. 
These models can be used to help to better understand earthaquakes,
their patterns and their mechanisms behind earthquake occurrences.



Recently, Evolutionary Computation has been used to learn risk models
using purely past earthquake occurrence as training data. While the
results were promising, we believe that a much better model could be
learned if domain knowledge, such as known theories and models on
earthquake distribution, were incorporated into the Evolutionary
Algorithm's training process.

In this work we approach this idea by improving former methods in two
ways: (1) We modify the genome representation of a model from an
area-based representation to an earthquake representation, and (2) we
use known methods from seismology (such as the Omori-Utsu formula) to 
refine the candidates generated by the GA.

We analyze the contributions from each of these modifications using
the methodologies described in the Collaboratory for The Study of
Earthquake Predictability (CSEP), and compare their performance with
(XXX method and YYY method). Our results indicate that (XXX result,
YYY result)
