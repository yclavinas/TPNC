\section{Conclusion}
\label{sec:conclusion}

In this paper we proposed three improvements to a Genetic Algorithm
for earthquake risk modeling. Pre-processing the data using Spectral
Clustering showed the greatest improvement, while simplifying the
representation reduced the number of parameters in the model without
reducing its quality. Using seismic equations, though showed less
successful.

These results show that there is a need for further ideas for the use
of meta-heuristics in modeling seismic processes. For example, the
Spectral Clustering step currently uses a trivial distance measure;
using a more appropriate modeling of space-time clustering of
earthquakes is one approach that shows promise.

Finally, many choices in this work regarding the target regions and
data filters aimed to focus our effort in ``regular'' scenarios of
earthquake occurrence. While we feel this is an appropriate decision
for an early work, we aim to remove these constraints from the data
set in the future, in order to see how the difficulty of the problem
affects the approaches used.
