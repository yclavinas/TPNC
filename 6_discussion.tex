\section{Discussion}
\label{sec:discussion}

From the results in Table~\ref{tab:loglikelihood}, we can see clearly
that using spectral-clustering to pre-process the data has a positive
effect on the quality of the final risk model. The analysis of the
effect on Figure~\ref{fig:Dunnet} also shows that models using the
reduced representation perform as well as the GAModel, with a much
smaller representation.

On the other hand, the use of the modified ETAS equation to generate
aftershocks in the model seems to have a negative effect on the
quality of the model. Whether this is because of bad parameters, or
due to poor modeling of the empirical equations is worthy of further
investigation. Figure~\ref{fig:heatmap} seems to suggest that the EMP
modification is overestimating the number of earthquakes, when
compared to the other models.

