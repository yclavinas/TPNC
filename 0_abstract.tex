Earthquake Risk Models describe the risk of occurrence of seismic
events on a given area based on information such as past earthquakes
in nearby regions and the seismic properties of the area under study.
These models can be used to help to better understand earthquakes,
their patterns and their mechanisms.

In previous work, we showed that Genetic Algorithms (GA) could
generate risk models with the same degree of precision as the Relative
Intensity (RI) method, which is considered a benchmark for this
problem.  However, a few shortcomings were also defined in that
approach: (1) The representation of the model in the Genetic Algorithm
was too sparse, (2) Domain knowledge was not used to create the model,
and (3) The relationship between foreshocks and aftershocks were not 
taken into account.

In this work, we try to address these three concerns. We propose a new
representation of a seismic risk model to be used as the genome of the
Genetic Algorithm. We introduces a hybrid model that incorporates
seismic theories about earthquake distribution (such as the Omori-Utsu
formula). And we use clustering to filter the earthquake catalog in
order to remove earthquakes that are likely to be aftershocks before
generating the risk model.

We examine each of these changes through simulations using the catalog
of Japanese earthquakes between 2000 and 2010. According to our
results, clustering the earthquake catalog produces better models,
while the proposed changes to representation did not show such a clear
effect. These results allow us to draw recommendations for future
developments.
