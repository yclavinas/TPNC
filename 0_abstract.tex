Earthquake Risk Models describe the risk of occurrence of seismic
events on an area based on information such as past earthquakes
in nearby regions and the seismic properties of the area under study.
These models can be used to help to better understand earthquakes,
their patterns and their mechanisms.

In previous work, we showed that Genetic Algorithms can be used to
generate earthquake risk models. However, we also noticed some
shortcomings in that approach. We propose three improvements to
address these shortcomings: A new representation that reduces the
search space, a hibridisation process that uses seismic equations to
generate the risk model from the GA representation using domain
knowledge, and a pre-processing step for the training data using
clustering.

We examine each of these changes through simulations using the catalog
of Japanese earthquakes between 2000 and 2010, and indicate the
contribution of each proposal to the quality of generated models.
