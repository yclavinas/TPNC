\section{Background}
\label{sec:background}

An Earthquake Risk model states the probability of earthquake
occurrence on a defined area and time period. These models are often
based on past occurrence of earthquakes (historical catalogs).  They
can also make use of seismic properties of the area under study, such
as faults, terrain properties, etc.

The ``prediction'' of earthquakes is a polemic subject, and no
research so far has come close to suggesting that individual large
scale earthquakes can be predicted. On the other hand, there is value
on the study of earthquake mechanisms and the generation of
statistical models of earthquake risk~\cite{Nature1999}.

In our previous work~\cite{ecta14}, we use a Genetic Algorithm (GA) to
optimise an Earthquake Risk Model, which is described in the framework
proposed by the Collaboratory for the Study of Earthquake
Predictability (CSEP).

The CSEP framework defines a model in reference to a geographical
region and a time period~\cite{zechar2010evaluating}. The geographical
region is divided in a grid, where each cell in the grid is called a
bin.

The model defines a number of expected earthquakes for each bin.  This
number must be a positive integer. A good model is one where the
number of estimated earthquakes in each bin corresponds to the actual
number of earthquakes that occurs in that bin during the target time
interval.

%%%%%%%%%%%%%%%%%%%%%%%%%%%%%%%%%%%%%%%%&&&&&&&&&&&&&&&&&&&&&&
\subsection{The GAModel}\label{sec:background:gamodel}

Using the CSEP framework described in the previous subsection, we
proposed the GAModel~\cite{ecta14}, which uses Genetic Algorithms to
generate an earthquake risk model based on earthquake catalog data.

In the GAModel, each individual is treated as a prediction in the CSEP
framework. The fitness of each individual will be calculated using the
log-likelihood of the catalog data given the individual's prediction.

\subsubsection*{Genome Representation and Evolutionary Operators.}

Each individual is represented as real valued array, where each
element is a bin, with an associated number of earthquakes. One-point
crossover, elitism and polynomial bounded mutation are used as
evolutionary operators. The relevant parameters were set as Elite Size
= 1, Crossover chance = 0.9, Mutation Chance = 0.1, Polynomial Bounded
parameters eta = 1, low = 0, up = 1.


\subsubsection*{Fitness Function and Selection.}

Let an individual $\Lambda = \{\lambda_1, \lambda_2, \ldots,
\lambda_N\}$ be a forecast in the CSEP framework. Let the set of
earthquake occurrences from the catalog be $\Omega = \{\omega_1,
\omega_2, \ldots, \omega_N\}$. The log-likelihood of the catalog data
given an individual is calculated as:

\begin{equation}
L(\Omega|\Lambda) = \sum_{i=1}^{n}L(\omega_i|\lambda_i) \\ =
\sum_{i=1}^{n} -\lambda_i + \omega_i\log\lambda_i - \log\omega_i! .
\end{equation}

To avoid overfitting, the period under consideration is divided into
sub-periods, the log-likelihood for each sub-period is calculated
separately, and the worst value is used as the fitness~\cite{ecta14}.

\subsection{Related Literature}

The usage of Evolutionary Computation (EC) in the field of earthquake
risk models is somewhat sporadic. Zhang and Wang~\cite{Zhang2012} used
Genetic Algorithms to fine tune an Artificial Neural Network (ANN) and
used this system to produce a forecast model. Zhou and
Zu~\cite{Feiyan2014} also proposed a combination of ANN and EC, but
their system forecasts only the magnitude parameter of
earthquakes. Sadat, in~\cite{sadat2015application}, used ANN and GA to
predict the magnitude of the earthquakes in North Iran.

%Fault Model parameters
There are more works when we discuss EC methods and estimation of
parameter values in seismological models. Nicknam et
al.~\cite{Nicknam2010} simulated some components from a seismogram
station and predicted seismograms for other stations. They combined
the Empirical Green’s Function (EGF) with GA. Kennett and
Sambridge~\cite{Kennett1992} used GA and associated teleseisms
procedures to determine the Fault Model parameters of an
earthquake.

%PGA
Another popular approach is to use EC methods do calculate the Peak
Ground Acceleration (PGA) parameter. The works done by Kerh et
al.~\cite{Kerh2010, Kerh2015} are a combination of ANN and GA to
estimate or predict PGA in Taiwan. Cabalar and
Cevik~\cite{Cabalar2009} work also aimed to predict the PGA, but their
work uses genetic programming (GP) and use strong-ground-motion data
from Turkey.

%Jafarian et al.~\cite{jafarian2010empirical}, used GP to develop an
%empirical predictive equation $v_max/a_max$ ratio of the shallow
%crustal strong ground motions recorded at free field sites. They found
%a relation between the $v_max/a_max$ and the earthquake magnitude and
%the source-to-site distance.
 
Ramos and Vázques~\cite{Ramos2011} used Genetic Algorithms to decide
the location of sensing stations. In this work they achieved, in
general, better results with the GA method when compared with the
Seismic Alert System (SAS) method and a greedy algorithm
method.

%Saeidian et al.~\cite{saeidian2016evaluation} work also based
%on the same idea of locating sensing stations. They do a comparison in
%performance between the GA and Bees Algorithm (BA) to decide which of
%those techniques would perform better when choosing the location of
%sensing stations. He found out that the GA was faster than the BA.

%Huda and Santosa \cite{ijse5762} published a paper in which the goal
%was to find, via GA, the speed of the waves P and S in the mantle and
%in the earth crust. P waves are indicated as the first fault found in
%seismological data and S waves are the changes caused in the phase of
%a P wave~\cite{ijse5762}. This work aimed to obtain a structure of the
%Japanese underground.

% TODO: No SPACE!
