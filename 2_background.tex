\section{Background}
\label{sec:background}

An Earthquake Risk model states the probability of earthquake
occurrence on a defined area and time period. These models are often
based on past ocurrence of earthquakes (historical catalogs).  They
can also make use of seismic properties of the area under study, such
as faults, terrain properties, etc.

The ``prediction'' of earthquakes is a polemic subject, and no
research so far has come close to suggesting that individual large
scale earthquakes can be predicted. On the other hand, there is value
on the study of earthquake mechanisms and the generation of
statistical models of earthquake risk~\cite{Nature1999}.

In our previous work~\cite{ecta14}, we use a Genetic Algorithm (GA) to
optimize an Earthquake Risk Model, which is described in the framework
proposed by the Collaboratory for the Study of Earthquake
Predictability (CSEP).

In the following subsections we describe both the CSEP framework and
the original GAModel. After that, we overview other relevant works
combining evolutionary approaches and the study of earthquakes.

\subsection{CSEP Forecast Framework}

The CSEP framework defines a model in reference to a geographical
region and a time period~\cite{zechar2010evaluating}. The geographical
region is divided in a grid, where each cell in the grid is called a
bin.

For example, in this paper we define the “Kanto” region as as the area
covered by latitude N34.8 to N37.05, and longitude E138.8 to
E141.05. This area is divided into 2025 bins (a grid of 45x45
squares).  Each bin has an area of approximately 25km$^2$

The model defines a number of expected earthquakes for each bin.  This
number must be a positive integer. A good model is one where the
number of estimated earthquakes in each bin corresponds to the actual
number of earthquakes that occurs in that bin during the target time
interval.


%%%%%%%%%%%%%%%%%%%%%%%%%%%%%%%%%%%%%%%%&&&&&&&&&&&&&&&&&&&&&&
\subsection{The GAModel}\label{sec:background:gamodel}

Using the CSEP framework described in the previous subsection, we
proposed the GAModel~\cite{ecta14}. The GAModel uses Genetic
Algorithms to generate an earthquake risk model based on earthquake
catalog data.

In the GAModel, each individual is a candidade Risk Model, which is
equivalent to a prediction $\Lambda = \{\lambda_1, \lambda_2, \ldots,
\lambda_n\}$ in the CSEP framework. The Genetic Algorithm will then
select the individuals based on the log-likelihood between the model
represented by the individual, and the catalog of earthquake
occurrences in the target location and time period.

\subsubsection*{Genome Representation}

In the GAModel, each indian individual is represented as real valued
array, where each element in the array is latitude/longitude bin in
the target area for the desired model. Each bin is associated to a
real value representing the earthquake risk in that location. For the
initial population, these values are drawn from an uniform
distribution between 0 and 1.

% TODO, maybe add the ``bins'' image from ECTA

During fitness evaluation, the risk value at each bin is converted to
an integer forecast, using a modification of the inverse poisson
function depicted in algorith~\ref{inversePoisson}. In this algorithm,
$x$ is the real value to be converted and $\mu$ is the mean of
earthquake observations across all bins in the catalog data.

\begin{algorithm}[H]\label{inversePoisson}
  \caption{Obtain a Poisson deviate from a $[0,1)$ value}
    \label{inversePoisson}
    \begin{algorithmic}
      \STATE $L \gets \exp{(-\mu)}, k \gets 1, prob \gets 1 * x$
      \WHILE{$prob > L$} 
      \STATE $k \gets k + 1$
      \STATE $prob \gets prob*x$
      \ENDWHILE
      \RETURN $k$
    \end{algorithmic}
\end{algorithm}

\subsubsection*{Fitness Function}

The GAModel uses the log-likelihood value between one individual
and the catalog data as its fitness function.

Let an individual $X = \{x_1, x_2, \ldots, x_N\}$, where $x_i$ is the
risk value associated with bin $i$, and $N$ is the total number of
bin. From $X$ we obtain the earthquake forecast $\Lambda =
\{\lambda_1, \lambda_2, \ldots, \lambda_N\}$ using
algorithm~\ref{inversePoisson}. This forecast is also the vector of
earthquake quantity expectations.

Let the set of earthquake ocurrences from the catalog be $\Omega =
\{\omega_1, \omega_2, \ldots, \omega_N\}$. The calculation of the
log-likelihood value for the $\omega_i$ observation with a given
expectation $\lambda$ is defined as:

\begin{equation}
L(\omega_i|\lambda_i) = -\lambda_i + \omega_i\log\lambda_i - \log\omega_i!
\end{equation}

The joint probability is the product of the likelihood of each bin, so
the log likelihood $L(\Omega|\Lambda)$ for the entire vector is the
sum of $L(\omega_i|\lambda_i)$ for every bin $i$:

\begin{equation}
L(\Omega|\Lambda) = \sum_{i=1}^{n}L(\omega_i|\lambda_i) \\ =
\sum_{i=1}^{n} -\lambda_i + \omega_i\log\lambda_i - \log\omega_i!
\end{equation}

Finally, this value is calculated for each year that composes the
training data, using the ``time slices'' method described
in~\cite{ecta14}.


\subsubsection*{Evolutionary Operators}\label{gaOperators}

GAModel uses One Point Crossover, Polynomial Bounded Mutation,
Tournament Selection and Elitism. The chance for using crossover and
mutation operators is tested independently for each individual in the
new population.

Valuse for the parameters for these operators are listed in
Table~\ref{GAParameters5.1}. These values are generally the same as
used in~\cite{ecta14}.

\begin{table}[H]
  \caption{Parameters used in the GAModel}
  \label{GAParameters5.1}
  \begin{center}
    \begin{tabular}{|l|r|}
      \hline
      Population Size & 500\\
      Generation Number & 100\\
      Elite Size & 1\\
      Tournament Size & 3\\
      Crossover Chance & 0.9\\
      Mutation Chance (individual) & 0.1\\
      Polynomial Bounded parameters & eta = 1, low = 0, up = 1\\
      \hline    
    \end{tabular}
  \end{center}
\end{table}

\subsection{Related Literature}

The usage of Evolutionary Computation (EC) in the field of earthquake
risk models is somewhat sporadic. Zhang and Wang~\cite{Zhang2012} used
Genetic Algorithms to fine tune an Artificial Neural Network (ANN) and
used this system to produce a forecast model. Zhou and
Zu~\cite{Feiyan2014} also proposed a combination of ANN and EC, but
their system forecasts only the magnitude parameter of
earthquakes. Sadat, in~\cite{sadat2015application}, used ANN and GA to
predict the magnitude of the earthquakes in North Iran.

%Fault Model parameters
There are more works when we discuss EC methods and estimation of
parameter values in seismological models. Nicknam et
al.~\cite{Nicknam2010} simulated some components from a seismogram
station and predicted seismograms for other stations. They combined
the Empirical Green’s Function (EGF) with GA. Kennett and
Sambridge~\cite{Kennett1992} used GA and associated teleseisms
procedures to determine the Fault Model parameters of an
earthquake. By doing so, they demonstrated that non-linear inversion
can be achieved for teleseismic problems without any calculation of
waves travel times.

%PGA
Another popular approach is to use EC methots do calculate the Peak
Ground Acceleration (PGA) parameter. The works done by Kerh et
al.~\cite{Kerh2010, Kerh2015} are a combination of ANN and GA to
estimate or predict PGA in Taiwan. Their goal was to decide which
areas may be considered potentially hazardous areas and they focused
on urban areas. They also stated that PGA is inversely proportional to
epicentre distance. Cabalar and Cevik~\cite{Cabalar2009} work also
aimed to predict the PGA, but their work uses genetic programming (GP)
and use strong-ground-motion data from Turkey.

Jafarian et al.~\cite{jafarian2010empirical}, used GP to develop an
empirical predictive equation $v_max/a_max$ ratio of the shallow
crustal strong ground motions recorded at free field sites. They found
a relation between the $v_max/a_max$ and the earthquake magnitude and
the source-to-site distance.
 
Ramos and Vázques~\cite{Ramos2011} used Genetic Algorithms to decide
the location of sensing stations. In this work they achieved, in
general, better results with the GA method when compared with the
Seismic Alert System (SAS) method and a greedy algorithm
method. Saeidian et al.~\cite{saeidian2016evaluation} work also based
on the same idea of locating sensing stations. They do a comparison in
performance between the GA and Bees Algorithm (BA) to decide which of
those techniques would perform better when choosing the location of
sensing stations. He found out that the GA was faster than the BA.

Huda and Santosa \cite{ijse5762} published a paper in which the goal
was to find, via GA, the speed of the waves P and S in the mantle and
in the earth crust. P waves are indicated as the first fault found in
seismological data and S waves are the changes caused in the phase of
a P wave~\cite{ijse5762}. This work aimed to obtain a structure of the
Japanese underground.
