\section{Bibliography and GAModel review}\label{estadoArte}
In this Section we will briefly explain Genetic Algorithms and then discuss some reports of the application of Evolutionary Computation and related method for Earthquake Risk Analysis and Review the GAModel.

\subsection{Evolutionary Computation and Earthquake Risk Analysis}

The usage of Evolutionary Computation (EC) in the field of earthquake risk models is somewhat sporadic. Zhang and Wang~\cite{Zhang2012} used Genetic Algorithms to fine tune an Artificial Neural Network (ANN) and used this system to produce a forecast model. Zhou and Zu~\cite{Feiyan2014} also proposed a combination of ANN and EC, but their system forecasts only the magnitude parameter of earthquakes. Sadat, in~\cite{sadat2015application}, used ANN and GA to predict the magnitude of the earthquakes in North Iran.

%Fault Model parameters
There are more works when we discuss EC methods and estimation of parameter values in seismological models. Nicknam et al.~\cite{Nicknam2010} simulated some components from a seismogram station and predicted seismograms for other stations. They combined the Empirical Green’s Function (EGF) with GA. Kennett and Sambridge~\cite{Kennett1992} used GA and associated teleseisms procedures to determine the Fault Model parameters of an earthquake. By doing so, they demonstrated that non-linear inversion can be achieved for teleseismic problems without any calculation of waves travel times. 


%PGA
The following works relate EC methods and the Peak Ground Acceleration (PGA) parameter. The works done by Kerh et al.~\cite{Kerh2010, Kerh2015} are a combination of ANN and GA to estimate or predict PGA in Taiwan. Their goal was to decide which areas may be considered potentially hazardous areas and they focused on urban areas. They also stated that PGA is inversely proportional to epicentre distance. Cabalar and Cevik~\cite{Cabalar2009} work also aimed to predict the PGA, but their work uses genetic programming (GP) and use strong-ground-motion data from Turkey.

Jafarian et al.~\cite{jafarian2010empirical}, used GP to develop an empirical predictive equation $v_max/a_max$ ratio of the shallow crustal strong ground motions recorded at free field sites. They found a relation between the $v_max/a_max$ and the earthquake magnitude and the source-to-site distance.
 
Ramos and Vázques~\cite{Ramos2011} used Genetic Algorithms to decide the location of sensing stations. In this work they achieved, in general, better results with the GA method when compared with the Seismic Alert System (SAS) method and a greedy algorithm method. Saeidian et al.~\cite{saeidian2016evaluation} work also based on the same idea of locating sensing stations. They do a comparison in performance between the GA and Bees Algorithm (BA) to decide which of those techniques would perform better when choosing the location of sensing stations. He found out that the GA was faster than the BA.

Huda and Santosa \cite{ijse5762} published a paper in which the goal was to find, via GA, the speed of the waves P and S in the mantle and in the earth crust. P waves are indicated as the first fault found in seismological data and S waves are the changes caused in the phase of a P wave~\cite{ijse5762}. This work aimed to obtain a structure of the Japanese underground.


%TODO: Last part in somewhat bad
%% CLAUS: A paper to CEC does not need a ``What is EC'' section
%\subsection{What is Evolutionary Computation And What are Genetic Algorithms}\label{EC}
%Evolutionary Computation (EC) is concerned with algorithms based on
%the Darwinian principles of natural selection. It may find, by try
%trial and error and based on a great amount of data, better solutions
%for problems that human beings may not find it easy to
%solve~\cite{koza2003genetic}. That could also be done without any
%domain knowledge about the problem to be
%controlled~\cite{Michie94machinelearning}.
%	
%The main goal of a Genetic Algorithm (GA), a EC technique, is to find
%approximated solutions in problems of search and optimization. Based
%on Koza~\cite{koza2003genetic}, GA are mechanism of search based on
%natural selection and genetic. They explore historical data to find
%optimum search points with some performance
%increment~\cite{Goldberg:1989:GAS:534133}.
%
%A GA uses those mechanisms to generate solutions to optimization and
%search problems. The first step is to create an initial population, a
%group of possible solutions, where each solution is called an
%individual. Those individuals have its fitness value estimated by a
%given function and those with greater fitness value are then chosen to
%reproduction. After some evaluations, we expect to find an optimum
%solution.
%
%Frequently, the initial population is randomly generated once it is
%common to ignore the main aspects that influence the algorithm
%performance. In other words, hence it is common to lack domain
%knowledge, the random population is a good way to start searching for
%optimum solutions.

The ReducedGAModel, which represents the idea of changing the
candidates solution representation (see Section~\ref{intro}), is a
method with a similar description of the GAModel. The difference is
that, in the ReducedGAModel each possible solution represents only a
fraction of the forecast where we expect to find specific risk areas.

%%YURI: 
%TODO: I didnt check this part yet. Review equations
\subsection{GAModel}\label{GAModel}
The GAModel is based on the framework suggested by the CSEP. The CSEP framework suggests that a forecast model uses a gridded rate
forecast~\cite{zechar2010evaluating}. The forecast will estimate the risk of earthquake occurrence in this target region, during the a time interval. We considered the target time interval of one year~\cite{ecta14}.

This gridded rate forecast is defined space-rate-time forecast and is divided in bins. Each bin represents a geographical interval. The whole target area of study is covered by a group of bins where each bin has an earthquake forecast value. This groups of bin represent the $\mu(x,y)$, the background intensity~\cite{zhuang2004analyzing}. In the GAModel, each possible solution is represented as an entire forecast model.

Therefore, the GAModel is considered as one method to generate
space-rate-time forecasts. It also could be described as:

\begin{equation}\label{gamodel}
\Lambda(t,x,y,M|\Upsilon_t) = \mu(x,y)
\end{equation}
where you can denote the number of earthquakes forecast in all bins 
as $\Lambda(t,x,y)$~\cite{zechar2010evaluating} given that $\Upsilon_t$ 
is the earthquake observation data up to time $t$.

\subsubsection{Genome Representation}\label{genomeGA}
In the GAModel each individual represents an entire forecast model. A individual is a real valued array X, where each element corresponds to one bin in the desired model (the number of bins n is defined by the region considered).

The values are sampled from the interval $[0,1)$. These real values are converted to a integer forecast, by a modification of the Poisson deviates extraction algorithm~\ref{inversePoisson} used in~\cite{ecta14}. In it $x$ is the real value that will be converted and $\mu$ is the mean of the earthquakes observations in the real data.

\begin{algorithm}[H]\label{inversePoisson}
	\caption{Obtain a Poisson deviate from a $[0,1)$ value}
	\begin{algorithmic}
		\STATE $L \gets \exp{(-\mu)}, k \gets 1, prob \gets 1 * x$
		\WHILE{$prob > L$} 
		\STATE $k \gets k + 1$
		\STATE $prob \gets prob*x$
		\ENDWHILE
		\RETURN $k$
	\end{algorithmic}
\end{algorithm}

The initial population, these values are sampled from a uniform distribution and they are randomly generated. For more details of the genome representation, please refer to~\cite{ecta14}.

We show an example to clarify how the GAModel works. The Kanto region, one of the four areas used in both studies, is divided into 2025 bins (a grid of 45x45 squares). The GAModel calculates an expected number of earthquakes for every bin on a determined time interval, so the GA searches for good values in 2025 bins.
\subsubsection{Fitness Function}\label{fitGA}

The GAModel uses the log-likelihood value, as fitness
function. The fittest individual among all the others of a generation, is maintained to the next, to guarantee that the best solution of the present generation is at least as good as the one from former generation. The bins $b_n$, define the set $\beta$ and $n$ is the size of the set $\beta$:
\begin{equation} 
\beta := {b_1,b_2,...,b_n},n = |\beta|.
\end{equation}
The probability values of the model $j$, expressed by the symbol
$\Lambda$, is made of expectations $\lambda_i^j$ by bin $b_i$. The
vector is define as:

\begin{equation}
\Lambda^j = 
\begin{pmatrix}
\lambda_1^j, 
\lambda_2^j, 
\hdots,
\lambda_i^j
\end{pmatrix}
;\lambda_i^j := \lambda_i^j(b_i),b_i \in \beta
\end{equation}

The vector of earthquake quantity expectations is defined as:
earthquake by time. The $\Omega$ vector is composed by observations
$\omega_i$ per bin $b_i$, as the $\Lambda$ vector:

\begin{equation}
\Omega = 
\begin{pmatrix}
\omega_1,
\omega_2,
\hdots,
\omega_i
\end{pmatrix}
;\omega_i =\omega_i(b_i),b_i \in \beta
\end{equation}

The calculation of the log-likelihood value for the $\omega_i$
observation with a given expectation $\lambda$ is defined as:


\begin{equation}
L(\omega_i|\lambda_i^j) = -\lambda_i^j + \omega_i\log\lambda_i^j - \log\omega_i!
\end{equation}

The joint probability is the product of the likelihood of each bin, so the logarithm $L(\Omega|\Lambda^j)$ is the sum of for
$L(\omega_i|\lambda_i^j)$ every bin $b_i$:

\begin{equation}\label{log-like}
\begin{split}
L^j = L(\Omega|\Lambda^j) = \sum_{i=1}^{n}L(\omega_i|\lambda_i^j)  \\
= \sum_{i=1}^{n} -\lambda_i^j + \omega_i\log\lambda_i^j - \log\omega_i!  
\end{split}
\end{equation}

\subsubsection{Evolutionary Operators}\label{gaOperators}
The GAModel use a combination of operators made available by the Distributed Evolutionary Algorithms in Python (DEAP)~\cite{DeRainville}. We used the One Point Crossover, the Polynomial Bounded Mutation, the Tournament Selection and Elitism. The parameters are described in the Table~\ref{GAParameters5.1}.

\begin{table}[H]
	\caption{Parameters used in GAModel and Emp-GAModel}
	\label{GAParameters5.1}
	\begin{center}
		\begin{tabular}{|l|r|}
			\hline
			Population Size & 500\\
			Generation Number & 100\\
			Elite Size & 1\\
			Tournament Size & 3\\
			Crossover Chance & 0.9\\
			Mutation Chance (individual) & 0.1\\
			Polynomial Bounded parameters & eta = 1, low = 0, up = 1\\
			\hline    
		\end{tabular}
	\end{center}
\end{table}

%The parameters of the Polynomial Bounded mutation function are: 
%\begin{enumerate}
%	\item eta = 1. Crowding degree of the mutation. A high eta will produce a mutant resembling its parent, while a small eta will produce a solution much more different;
%	\item low = 0. The lower bound of the search space;
%	\item up = 1. The upper bound of the search space.
%\end{enumerate}

The chance of applying both mutation and crossover operators takes into account only their own chance of occurrence. This means that it may be the case that one of them or both are not applied.

